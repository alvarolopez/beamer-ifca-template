\documentclass{beamer}

\usetheme{ifca}


%\usepackage[spanish]{babel}

\usepackage[latin1]{inputenc}

\usepackage{lmodern}
\usepackage[T1]{fontenc}

\usepackage{wrapfig}

% Or whatever. Note that the encoding and the font should match. If T1
% does not look nice, try deleting the line with the fontenc.


\title[Ejemplo de Presentacion] % (optional, use only with long paper titles)
{Ejemplo de presentaci�n con el tema del IFCA}

\subtitle{}

\author[A. L�pez Garc�a] % (optional, use only with lots of authors)
{�lvaro L�pez Garc�a \inst{1}}
% - Give the names in the same order as the appear in the paper.
% - Use the \inst{?} command only if the authors have different
%   affiliation.

\institute[aa] % (optional, but mostly needed)
{
  \inst{1}%
  Instituto de F�sica de Cantabria - CSIC-UC\\
  Advanced Computing and e-Science Group}
% - Use the \inst command only if there are several affiliations.
% - Keep it simple, no one is interested in your street address.

\date[] % (optional, should be abbreviation of conference name)
{Seminarios Tecnicos RES\\
IFCA\\
May 2010 - Santander, Spain}

% - Either use conference name or its abbreviation.
% - Not really informative to the audience, more for people (including
%   yourself) who are reading the slides online

\subject{}
% This is only inserted into the PDF information catalog. Can be left
% out.



% If you have a file called "university-logo-filename.xxx", where xxx
% is a graphic format that can be processed by latex or pdflatex,
% resp., then you can add a logo as follows:
%
%\pgfdeclareimage[height=1cm]{ifca-logo}{images/IFCAlogo}
%\pgfdeclareimage[height=1cm]{csic-logo}{images/csic}
%\pgfdeclareimage[height=1cm]{uc-logo}{images/UNICANlogo}
%
%\logo{
%    \pgfuseimage{ifca-logo}
%    \pgfuseimage{csic-logo}
%    \pgfuseimage{uc-logo}
%    \vspace{-5pt}
%}
%
%\def\colorize<#1>{
%    \temporal<#1>
%        {\color{black!50}}
%        {\color{red!50}}
%        {\color{black!50}}
%}
%
%\setbeamertemplate{navigation symbols}{} %no nav symbols


% Delete this, if you do not want the table of contents to pop up at
% the beginning of each subsection:
\AtBeginSubsection[]
{
  \begin{frame}<beamer>{Outline}
    \tableofcontents[currentsection,currentsubsection]
  \end{frame}
}


% If you wish to uncover everything in a step-wise fashion, uncomment
% the following command:

%\beamerdefaultoverlayspecification{<+->}


\begin{document}


\begin{frame}
  \titlepage
\end{frame}

\begin{frame}{Outline}
  \tableofcontents
  % You might wish to add the option [pausesections]
\end{frame}


\section{Section}

\subsection{Subsection}

\begin{frame}
    \frametitle{Slide}
    \framesubtitle{Slide subtitle}
    This is an empty slide
\end{frame}

\begin{frame}{The end}
  \begin{center}
    \huge
    �Thank you very much!
  \end{center}
\end{frame}

% All of the following is optional and typically not needed.
%\appendix
%\section<presentation>*{\appendixname}
%\subsection<presentation>*{For Further Reading}
%
%\begin{frame}[allowframebreaks]
%  \frametitle<presentation>{For Further Reading}
%
%  \begin{thebibliography}{10}
%
%  \beamertemplatebookbibitems
%  % Start with overview books.
%
%  \bibitem{Author1990}
%    A.~Author.
%    \newblock {\em Handbook of Everything}.
%    \newblock Some Press, 1990.
%
%
%  \beamertemplatearticlebibitems
%  % Followed by interesting articles. Keep the list short.
%
%  \bibitem{Someone2000}
%    S.~Someone.
%    \newblock On this and that.
%    \newblock {\em Journal of This and That}, 2(1):50--100,
%    2000.
%  \end{thebibliography}
%\end{frame}

\end{document}


